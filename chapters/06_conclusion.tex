\chapter{Conclusion}

In this work the state of the art scalable placement and routing algoithm ortho was extended with three signal distribution networks, each adding different functionality to the preprocessing and base algorithm.\\
The ordering distribution network aims to reduce wire crossings and area, through the ordering of PIs and allowing gates to be placed in the input area. The results showed that not only these goals could be reached but also suggests that the ordering of primary inputs affects the placement and routing of the whole logic network resulting in even more benefits.\\
The majority distribution network aims to show the difficulty of applying one of the theoretically biggest advantage of QCA technology, the placement and routing of majority gates. Although the number of gates could be reduced significantly in some networks, the implementation into ortho results in higher area usage and wiring effort. Big part of this is the need for buffer insertion, resulting from the 2DDWave scheme used within ortho in combination of the signal delay inflicted by the placement of majority gates. For further improvement, the area usage could be reduced by finding a way to place majority gates in the same rows or column, which would however penalize the running time. Also the use of an algorithm which naturally supports a clocking, in which majority gates can be implemented are suggested to not improve the placement and routing of majority gates.\\
The sequential distribution network qualifies the ortho algorithm to design sequential circuits after rethinking the basic theory of sequentiality, storage elements and clocking for the QCA domain. Therefore, so called wire registers were proposed and used in the placement and routing. This has to be considered for the Bennet clocking and slows down the circuit quite heavily, depending on the size of the layouts combinational part. The ideas of the wire register were even powerful enough to implement a RAM cell based on a 2:1 mux and are suggested to be used for further research in implementing sequential behavior into QCA circuits.