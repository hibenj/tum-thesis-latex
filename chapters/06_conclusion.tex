\chapter{Conclusion}

In this work, the state-of-the-art scalable placement and routing algorithm, ortho, was extended with three signal distribution networks, each adding distinct functionality to the preprocessing and base algorithm.

The ordering distribution network aims to reduce wire crossings and layout area by ordering primary inputs and allowing gates to be placed in the input area. The results demonstrate that not only were these goals achieved, but the ordering of primary inputs also affected the placement and routing of the entire logic network resulting in additional benefits.

The majority distribution network aims to demonstrate the difficulty of applying one of the theoretical advantages of QCA technology, the placement and routing of majority gates. Although the number of gates could be reduced significantly in some networks, the implementation into ortho resulted in higher layout area usage and wiring effort. This was largely due to the need for buffer insertion, stemming from the 2DDWave scheme used within ortho in combination with the signal delay caused by the placement of majority gates. To improve this, reducing layout area usage by placing majority gates in the same rows or columns could be considered, at the expense of increased running time. Additionally, using an algorithm that naturally supports clocking, in which majority gates can be implemented, is suggested as a solution to improve the placement and routing of majority gates.

The sequential distribution network enhances ortho's ability to design sequential circuits by rethinking the basic theory of sequentiality, storage elements, and clocking for the QCA domain. Wire registers were proposed and used in the placement and routing, and this has to be considered for the Bennett clocking and slows down the circuit, depending on the size of the layout's combinational part. The concepts of wire registers were powerful enough to implement a RAM cell based on a 2:1 MUX and are suggested for further research in implementing sequential behavior into QCA circuits.