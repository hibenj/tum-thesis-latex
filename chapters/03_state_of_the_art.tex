\chapter{State of the Art}\label{chapter:SotA}
In this chapter various approaches trying to solve the placement and routing problem for QCA are reviewed. In the first part algorithms, which are able to work only with combinational circuits, are investigated under the theoretical groundwork done in chapter \ref{chapter:Preliminaries}. Thereby both algorithms determining \textit{optimal} and those who determine \textit{scalable} solutions exist. In the second part ideas and challenges of sequential placement and routing algorithms are investigated.

\section{Combinational P\&R Algorithms}

In section \ref{sec:PR} the placement and routing problem was stated to be $\mathcal{NP}$-hard. This means, that on a deterministic Touring Machine there exists no algorithm that can determine a solution in polynomial time, but the proofs of solutions with an answer "yes" can be verified on it. This can be written as $\mathcal{P} \neq \mathcal{NP}$. In order to enable optimal solutions for the placement and routing algorithms so called \textit{Satisfiability Solvers} are used. The general principle can be seen in figure !!. First an instance of a satisfiability problem is encoded from a problem instance in such a way that they are equisatisfiable. This means that if there exists a solution to the satisfiability problem instance there also exists a solution to the original problem instance. In the second step, the satisfiability instance is passed to a specialized solver
returning an assignment to each variable in the encoding if such a solution exists or otherwise UNSAT. Lastly a solution to the original problem can be derived from the assignment \cite{Walter}. It has to be mentioned that these solvers even though they are improving average case performance drastically, they are still bound to the complexity barrier of their underlying system, meaning  $\mathcal{P} \neq \mathcal{NP}$ \cite{NP-P}.\\
With this knowledge optimal placement and routing algorithms can be discussed. Two approaches from \cite{Walter} are reviewed for this. The first algorithm "Exact Placement and Routing" finds a valid placement, routing and clocking, also described as $(p, r, c)$, given an empty layout $L$ and a logic network $N$. In order to find an optimal solution, the minimum layout size $w \times h$
has to be determined for which the constraints of $(p, c, r)$ hold true. Therefore all possible sizes of layouts are encoded and passed to a satisfiability solver iteratively and the first layout for which the solver returns true is the minimum or rather optimal solution. The satisfiability solver utilized by this algorithm is a \textit{boolean satisfiability}-solver (SAT-solver), which is sufficient information for this work. The experimental results show that the determined layouts of the algorithm are many times smaller than the compared state of the art CITE. But due to the complexity of the algorithm utilizing satisfiability solvers, the algorithm times out for quite small circuits already, making it insufficient for the manufacturing of commercial QCA circuits.\\
In the book \cite{Walter} another exact P\&R algorithm is proposed. The idea is to create a \textit{one-pass synthesis}, which combines logic synthesis and physical design in a single run. Therefore this algorithm has to tackle two $\mathcal{NP}$-hard problems relying again on the power of satisfiability solvers. This particular algorithm uses a \textit{satisfiability modulo theories}-solver (SMT-solver). The idea is to eliminate some shortcomings of the two-step synthesis derived from CMOS. This includes treating wires as gates since the costs are equal in QCA and including data synchronization, which is dependent on the tiles passed. In this manner a SAT problem can be formed and passed to a SAT-solver. The instances are now created only passing a empty layout $L$ of size $w \times h$. Even though this algorithm is able to find \textit{truly minimal} solution since the non-optimal logic networks are eliminated, the experimental results show the same problems as in the exact P\&R approach. This means that the high complexity of the satisfiability solver leads to a time-out of the algorithms for circuits with a gate size $|N| \geq 30$.\\
These results lead to the usage of scalable placement and routing algorithms. These approaches trade optimality of the circuit, like layout size for computing time. This makes them scalable in the time domain and therefore applicable for the manufacturing o commercial QCA circuits. All these algorithms reviewed here are based on the original VLIS process, meaning they treat logic synthesis and physical design as their own problem.\\




Quick recap on the Algorithms from the book \\
Quick recap on ortho and why i selected ortho for my work.\\
Maybe point out that balancing is not used.

\section{Sequential P\&R Algorithms}
Present the ideas in papers for QCA standard cell placement and routing. \\
Point out why they are not actionable: Reasons like clocking or cells aren't producible, no automated algorithms