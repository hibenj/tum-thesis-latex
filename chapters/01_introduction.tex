% !TeX root = ../main.tex
% Add the above to each chapter to make compiling the PDF easier in some editors.

\chapter{Introduction}\label{chapter:introduction}
Quantum-dot Cellular Automata (QCA) is a promising technology for the design of ultra-low power, high-density, and high-performance digital circuits. QCA technology is based on the manipulation of the electronic states of quantum dots (QDs) to perform logic operations. The potential benefits of QCA include its ability to operate at extremely low power levels and high device density.

\section{Motivation}
Although the theory sounds very promising, the design of QCA circuits is a challenging task due to the complexity of the underlying physics and the lack of appropriate design tools. Placement and routing are two critical steps in the design of QCA circuits that determine the overall performance, power consumption, and area efficiency of the circuit.

Placement refers to the process of arranging the QCA cells on the chip. It involves determining the optimal position of the cells to minimize the number of interconnects and the routing area. Routing, on the other hand, refers to the process of connecting the QCA cells to form a functional circuit. It involves determining the optimal path for the interconnects to minimize the routing area. In QCA placement an routing are strongly related to each other and have to be viewed as a connected process. The placement and routing is a  non-trivial task due to the constraints imposed by the QCA technology, such as local and global timing constraints.

The importance of placement and routing in the design of QCA circuits cannot be overstated. The performance, power consumption, and area efficiency of the circuit are all directly affected by the placement and routing of the QCA cells. An optimal placement and routing can significantly improve the performance and reduce the power consumption of the circuit, making it more suitable for practical applications.

In this thesis, we will explore the various placement and routing techniques used in the design of QCA circuits. We will emphazise the need for scalable placement and routing algorithms and investigate the trade-offs between performance, power consumption, and area efficiency, and we will propose new techniques to improve the design of QCA circuits.

\section{Objective}
Hence, the limitations of the current state of the art scaling placement and routing algorithms should be overcome by introducing so called distribution networks. The proposed distribution networks add new functionalities to a scaling placement and routing algorithm, such as the ability to reduce wire crossings and layout area, place and route majority gates and sequential logic.

The first distribution network, called the Ordering Distribution Network, is designed to reduce wire crossings and layout area by introducing a new ordering of primary inputs (PIs). This improves the overall area required for the circuit and reduces the number of wire crossings and therefore the costs of the designed QCA circuits.

The second distribution network, called the Majority Gates Distribution Network, is designed to enable the placement of majority gates in the QCA circuit. However, it is important to note that while the number of gates placed is reduced, the layout area may actually increase due to the timing constraints.

The third distribution network, called the Sequential Distribution Network, enables the algorithm for the first time to place and route sequential logic in QCA. This is a significant advancement as current placement and routing algorithms are not able to handle sequential logic.

The main objective of this thesis is to improve the scalability and performance of the placement and routing algorithm for QCA circuits by introducing these new distribution networks. The proposed distribution networks have the potential to improve the scalability and performance of the algorithm while enabling the placement and routing of majority gates and sequential logic, which were previously not possible. This thesis aims to provide a deeper understanding of the challenges and opportunities of QCA technology and to contribute to the development of practical design tools for QCA circuits.

%\section{Section}
%Citation test~\parencite{latex}.
%
%\subsection{Subsection}
%
%See~\autoref{tab:sample}, \autoref{fig:sample-drawing}, \autoref{fig:sample-plot}, \autoref{fig:sample-listing}.
%
%\begin{table}[htpb]
%  \caption[Example table]{An example for a simple table.}\label{tab:sample}
%  \centering
%  \begin{tabular}{l l l l}
%    \toprule
%      A & B & C & D \\
%    \midrule
%      1 & 2 & 1 & 2 \\
%      2 & 3 & 2 & 3 \\
%    \bottomrule
%  \end{tabular}
%\end{table}
%
%\begin{figure}[htpb]
%  \centering
%  % This should probably go into a file in figures/
%  \begin{tikzpicture}[node distance=3cm]
%    \node (R0) {$R_1$};
%    \node (R1) [right of=R0] {$R_2$};
%    \node (R2) [below of=R1] {$R_4$};
%    \node (R3) [below of=R0] {$R_3$};
%    \node (R4) [right of=R1] {$R_5$};
%
%    \path[every node]
%      (R0) edge (R1)
%      (R0) edge (R3)
%      (R3) edge (R2)
%      (R2) edge (R1)
%      (R1) edge (R4);
%  \end{tikzpicture}
%  \caption[Example drawing]{An example for a simple drawing.}\label{fig:sample-drawing}
%\end{figure}
%
%\begin{figure}[htpb]
%  \centering
%
%  \pgfplotstableset{col sep=&, row sep=\\}
%  % This should probably go into a file in data/
%  \pgfplotstableread{
%    a & b    \\
%    1 & 1000 \\
%    2 & 1500 \\
%    3 & 1600 \\
%  }\exampleA
%  \pgfplotstableread{
%    a & b    \\
%    1 & 1200 \\
%    2 & 800 \\
%    3 & 1400 \\
%  }\exampleB
%  % This should probably go into a file in figures/
%  \begin{tikzpicture}
%    \begin{axis}[
%        ymin=0,
%        legend style={legend pos=south east},
%        grid,
%        thick,
%        ylabel=Y,
%        xlabel=X
%      ]
%      \addplot table[x=a, y=b]{\exampleA};
%      \addlegendentry{Example A};
%      \addplot table[x=a, y=b]{\exampleB};
%      \addlegendentry{Example B};
%    \end{axis}
%  \end{tikzpicture}
%  \caption[Example plot]{An example for a simple plot.}\label{fig:sample-plot}
%\end{figure}
%
%\begin{figure}[htpb]
%  \centering
%  \begin{tabular}{c}
%  \begin{lstlisting}[language=SQL]
%    SELECT * FROM tbl WHERE tbl.str = "str"
%  \end{lstlisting}
%  \end{tabular}
%  \caption[Example listing]{An example for a source code listing.}\label{fig:sample-listing}
%\end{figure}
