\chapter{\abstractname}

In this thesis, a new approach for placement and routing in Quantum-dot Cellular Automata (QCA) is presented by introducing Distribution Networks (DNs). These networks enhance the functionalities of an already existing scalable placement and routing algorithm. The first DN, called Ordering Distribution Network (ODN), reduces the number of wire crossings by an average of $13\%$ and the layout area by an average of $17\%$ by reorganizing primary inputs. The second DN, called Majority Gates Distribution Network (MGDN), enables the algorithm to place and route "+"-majority gates while satisfying timing constraints, resulting in an average of $865\%$ increase in layout area despite the fact that in average $72\%$ fewer gates are required for logic networks containing majority gates. Notably, the MGDN does not incorporate any wire crossings, thus reducing the number of wire crossings by and average of $40\%$ in the layout. Lastly, the third DN, called Sequential Distribution Network (SDN), is the first to enable the placement and routing of sequential logic in QCA. The proposed method is evaluated through extensive simulation and experimentation and shows different trade-offs in terms of design metrics like performance, layout area or number of wire crossings.

%Quantum-dot cellular automata (QCA) is a promising nanoscale technology, which overcomes well-known restrictions in the state of the art CMOS technology e.g. power dissipation or cost of lithography. In order to design QCA circuits the logic needs to placed and routed, requiring algorithms for a high number of gates. In this work the state of the art placement and routing algorithm \textit{ortho} is extended by three Signal Distribution Networks. A Ordering Network, which reduces the average layout area by $17\%$ and the average wire crossings by $13\%$. Also Majority Gate Network is introduced, allowing the placement of majority gates, resulting in an average area increase by a factor of over $8.5$. The third Distribution Network allows the placement and routing algorithm to place sequential circuits in QCA technology with rethought theory based on the timing constraints present in QCA.

%short channel effect, impurity variations, high cost of lithography and more importantly, the heat
%TODO: Abstract


